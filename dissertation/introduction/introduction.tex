% !TEX root = ../main.tex

\chapter{Introduction}

This chapter outlines the motivations behind the project and presents the objectives for the research made in this project.

\section{Objective}

The title of this project is \thesistitle. The first part of this title is 'An intelligent sentiment analysis system'. This indicates that some sort of system must be created which analyses sentiment proxy intelligently. This means that there must be an emphasis put within this project on the understanding sentiment proxy, and use this understanding to analyse sets of data which involve sentiment. This is the followed by 'for estimating price changes in financial markets'. This indicates that the previous understanding of sentiment proxies should be used to understand price changes in the market. This requires an understanding of the relationship between sentiment proxy and price changes. This must be heavily emphasised within the project. This understanding of the relationship may then be used to develop a model for estimating these changes once causation between these segments is understood. This assumes there is causation however. These items laid out form one overarching question: What is the relationship between sentiment and prices changes? This question can be broken down into two segments.
\begin{enumerate}
    \item How do we start understand the relationship between sentiment and price changes? Sentiment is an inherently qualitative form of data and price changes are an inherently quantitative form of data. Therefore how is qualitative data used in a quantitative setting?
    \item Once we can use sentiment in the same system as price changes, what is the relationship between them? Is there a causal link between the two items or are they even correlated?
\end{enumerate}

\section{Report Structure}

Chapter 2 outlines the background required to understand some of the decisions made in order to answer the posed questions. As well as this it contains some prerequisite knowledge required to understand some of the concepts within the report itself.

Chapter 3 outlines the design choices made within the creation of the system that was built in order to explore the questions that are put forward in this project.

Chapter 4 outlines the details of implementing the system laid out in chapter 4 in a manner that is useful.

Chapter 5 outlines the results provided by the implemented system in reference to the questions being explored, as well as exploring what these results might mean.

Chapter 6 reflects upon the work done within the project as well as how it may be enhanced.